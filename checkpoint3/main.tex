\documentclass[11pt]{article}
\usepackage[letterpaper, margin=1in]{geometry}
\usepackage{enumitem}
\usepackage{booktabs}
\usepackage{amsmath}
\usepackage{graphicx}
\usepackage{hyperref}

\title{Checkpoint 3 Progress Report\\
\large Efficient Computation of Crossing Components and Shortcut Hulls}
\author{Zohaib Aslam (za08134), M. Mansoor Alam (ma08322)}
\date{}

\begin{document}

\maketitle

\section{Implementation Summary}
We have successfully implemented all components of the polygon simplification algorithm described in the research paper. Our implementation includes:
\begin{itemize}[leftmargin=*, noitemsep]
  \item \textbf{Polygon and Edge Representation:} Data structures for simple polygons as ordered sets of points, with edges stored in topological order.
  \item \textbf{Edge Crossing Computation:} Adjacency lists $N^+(v)$ and $N^-(v)$ for each vertex $v$, sorted in descending order.
  \item \textbf{Crossing Component Computation:} Implementation of the pseudo-intersection graph $G_P$ approach instead of the full intersection graph $G_I$, with BFS for connected components.
  \item \textbf{Shortcut Hull Processing:} Convex hull computation and pocket segmentation for separate processing.
  \item \textbf{Optimal Shortcut Hull Computation:} Dynamic programming algorithm using the cost function $c(Q) = \lambda \cdot \beta(Q) + (1 - \lambda) \cdot \alpha(Q)$ where $\beta(Q)$ is perimeter, $\alpha(Q)$ is area, and $\lambda$ is a user parameter.
\end{itemize}

\section{Correctness Testing}
Our testing combines unit tests, integration tests, and visual validation:

\subsection{Unit Testing}
Each algorithm component was tested individually with specific test cases:
\begin{itemize}[leftmargin=*, noitemsep]
  \item \textbf{Edge Crossing Detection:} Geometric verification with cases like squares with crossing diagonals, complex polygons with known crossings, and edge cases (collinear edges, shared vertices and near-parallel edges).
  \item \textbf{Adjacency List Construction:} Validation against reference structures, verifying correct sorting \& empty lists.
  \item \textbf{Component Detection:} BFS traversal tested against expected connected components.
\end{itemize}

\subsection{Integration Testing}
We tested how components work together:
\begin{itemize}[leftmargin=*, noitemsep]
  \item Verified pseudo-intersection graph against full intersection graph on small test cases
  \item Tested shortcut hull computations with varying $\lambda$ values (0.0, 0.25, 0.5, 0.75, 1.0)
  \item End-to-end tests on polygons with known expected simplifications
\end{itemize}

\subsection{Visual Validation}
We implemented visualization tools:
\begin{itemize}[leftmargin=*, noitemsep]
  \item Original polygons with crossing edges highlighted
  \item Color-coded crossing components 
  \item Side-by-side comparisons of original polygons and shortcut hulls
  \item Lambda-effect visualizations
\end{itemize}

\subsection{Sample Test Cases - Here are two specific cases}
\begin{enumerate}[leftmargin=*, noitemsep]
  \item \textbf{Square with Crossing Diagonals:}
  \begin{itemize}[noitemsep]
    \item Input: Square with vertices at (0,0), (4,0), (4,4), (0,4)
    \item Shortcuts: (1,3), (2,4)
    \item Result: Algorithm correctly identified one crossing
  \end{itemize}
  
  \item \textbf{Star-shaped Polygon:}
  \begin{itemize}[noitemsep]
    \item Input: 10-vertex star polygon
    \item Result: Algorithm correctly identified 3 components
  \end{itemize}
\end{enumerate}

\section{Complexity and Runtime Analysis}

\subsection{Theoretical Analysis}
The algorithm achieves improved performance across its core components. For \textbf{Edge Crossing Detection}, the time complexity is $\mathcal{O}(n + m + k)$, where $n$ is the number of vertices, $m$ the number of potential shortcut edges, and $k$ the number of edge crossings. This efficiency is enabled by a column-wise sweep algorithm and adjacency list structures, avoiding the naive $\mathcal{O}(m^2)$ approach. \textbf{Component Computation} runs in $\mathcal{O}(\min(n + m + k, n^2))$ by leveraging a pseudo-intersection graph and performing BFS, which is significantly better than the naive $\mathcal{O}(n^4)$ method. Finally, the \textbf{Shortcut Hull Dynamic Programming} component operates in $\mathcal{O}(h\chi^3 + \chi n)$, where $h$ is the number of crossing components and $\chi$ is the maximum size of any component. By focusing on component hierarchies instead of the full polygon, the algorithm reduces the search space and optimizes pathfinding.

\subsection{Empirical Analysis}
Our benchmarking methodology involved:
\begin{itemize}[leftmargin=*, noitemsep]
  \item Polygons with controlled complexity (100 to 10,000 vertices)
  \item Multiple runs per size to account for variance
  \item Measurement of wall-clock time and CPU cycles
\end{itemize}

Benchmark results confirm sub-quadratic scaling:

\begin{table}[h]
\centering
\begin{tabular}{ccc}
\toprule
Polygon Size & Runtime (s) & Shortcut Hull Size \\
\midrule
100 & 0.027 & 20 \\
200 & 0.068 & 28 \\
400 & 0.162 & 41 \\
800 & 0.425 & 58 \\
1000 & 0.601 & 64 \\
2000 & 1.362 & 90 \\
4000 & 2.457 & 128 \\
10000 & 3.874 & 203 \\
\bottomrule
\end{tabular}
\end{table}

The runtime grows smoothly and remains practical even for large inputs (under 4 seconds for 10,000 vertices), validating our implementation approach.

\section{Baseline and Comparative Evaluation}
We compared against three baseline approaches:

\begin{enumerate}[leftmargin=*, noitemsep]
  \item \textbf{Naive Edge Crossing Detection:}
  \begin{itemize}[noitemsep]
    \item For $n=1000$, our optimized algorithm: 0.601s
    \item Naive implementation: 7.823s (13x slower)
  \end{itemize}
  
  \item \textbf{Greedy Shortcut Selection:}
  \begin{itemize}[noitemsep]
    \item Our DP-based approach produces solutions with 18\% lower total cost on average
    \item For complex pocket structures, improvement increases to 25-30\%
  \end{itemize}
  
  \item \textbf{Douglas-Peucker Algorithm:}
  \begin{itemize}[noitemsep]
    \item Our algorithm better preserves topological properties
    \item Our approach balances perimeter and area optimization through $\lambda$
    \item Douglas-Peucker is faster but produces less optimal results on our cost function
  \end{itemize}
\end{enumerate}

\section{Challenges and Solutions}
\begin{itemize}[leftmargin=*, noitemsep]
  \item \textbf{Algorithm Complexity:} The algorithm has multiple complex steps and data structures.
  \begin{itemize}[noitemsep]
    \item \emph{Solution:} Used a modular approach, breaking implementation into small, testable chunks.
  \end{itemize}
  
  \item \textbf{Inconsistency in Shortcut Hull Results:} Some polygons yield incorrect results.
  \begin{itemize}[noitemsep]
    \item \emph{Solution:} Currently analyzing code for bugs, comparing with the paper's implementation.
  \end{itemize}
\end{itemize}

\section{Enhancements}
\begin{itemize}[leftmargin=*, noitemsep]
  \item \textbf{Testing on Different Dataset:} Using custom dataset with polygons of up to 10,000 vertices for comprehensive testing.
  
  \item \textbf{Visualization Tools:} Developed additional tools:
  \begin{itemize}[noitemsep]
    \item Lambda-effect visualizer
    \item Component coloring
    \item Side-by-side comparison of original polygon, convex hull, and shortcut hull
  \end{itemize}
  
  \item \textbf{Pocket Detection Optimization:} Implemented:
  \begin{itemize}[noitemsep]
    \item More Robust equality checking for floating-point coordinates
    \item Optimized search for hull vertices \& Early termination for trivial pockets
  \end{itemize}
\end{itemize}

\end{document}